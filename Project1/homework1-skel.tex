
    




    
\documentclass[11pt]{article}

    
    \usepackage[breakable]{tcolorbox}
    \tcbset{nobeforeafter} % prevents tcolorboxes being placing in paragraphs
    \usepackage{float}
    \floatplacement{figure}{H} % forces figures to be placed at the correct location
    
    \usepackage[T1]{fontenc}
    % Nicer default font (+ math font) than Computer Modern for most use cases
    \usepackage{mathpazo}

    % Basic figure setup, for now with no caption control since it's done
    % automatically by Pandoc (which extracts ![](path) syntax from Markdown).
    \usepackage{graphicx}
    % We will generate all images so they have a width \maxwidth. This means
    % that they will get their normal width if they fit onto the page, but
    % are scaled down if they would overflow the margins.
    \makeatletter
    \def\maxwidth{\ifdim\Gin@nat@width>\linewidth\linewidth
    \else\Gin@nat@width\fi}
    \makeatother
    \let\Oldincludegraphics\includegraphics
    % Set max figure width to be 80% of text width, for now hardcoded.
    \renewcommand{\includegraphics}[1]{\Oldincludegraphics[width=.8\maxwidth]{#1}}
    % Ensure that by default, figures have no caption (until we provide a
    % proper Figure object with a Caption API and a way to capture that
    % in the conversion process - todo).
    \usepackage{caption}
    \DeclareCaptionLabelFormat{nolabel}{}
    \captionsetup{labelformat=nolabel}

    \usepackage{adjustbox} % Used to constrain images to a maximum size 
    \usepackage{xcolor} % Allow colors to be defined
    \usepackage{enumerate} % Needed for markdown enumerations to work
    \usepackage{geometry} % Used to adjust the document margins
    \usepackage{amsmath} % Equations
    \usepackage{amssymb} % Equations
    \usepackage{textcomp} % defines textquotesingle
    % Hack from http://tex.stackexchange.com/a/47451/13684:
    \AtBeginDocument{%
        \def\PYZsq{\textquotesingle}% Upright quotes in Pygmentized code
    }
    \usepackage{upquote} % Upright quotes for verbatim code
    \usepackage{eurosym} % defines \euro
    \usepackage[mathletters]{ucs} % Extended unicode (utf-8) support
    \usepackage[utf8x]{inputenc} % Allow utf-8 characters in the tex document
    \usepackage{fancyvrb} % verbatim replacement that allows latex
    \usepackage{grffile} % extends the file name processing of package graphics 
                         % to support a larger range 
    % The hyperref package gives us a pdf with properly built
    % internal navigation ('pdf bookmarks' for the table of contents,
    % internal cross-reference links, web links for URLs, etc.)
    \usepackage{hyperref}
    \usepackage{longtable} % longtable support required by pandoc >1.10
    \usepackage{booktabs}  % table support for pandoc > 1.12.2
    \usepackage[inline]{enumitem} % IRkernel/repr support (it uses the enumerate* environment)
    \usepackage[normalem]{ulem} % ulem is needed to support strikethroughs (\sout)
                                % normalem makes italics be italics, not underlines
    \usepackage{mathrsfs}
    

    
    % Colors for the hyperref package
    \definecolor{urlcolor}{rgb}{0,.145,.698}
    \definecolor{linkcolor}{rgb}{.71,0.21,0.01}
    \definecolor{citecolor}{rgb}{.12,.54,.11}

    % ANSI colors
    \definecolor{ansi-black}{HTML}{3E424D}
    \definecolor{ansi-black-intense}{HTML}{282C36}
    \definecolor{ansi-red}{HTML}{E75C58}
    \definecolor{ansi-red-intense}{HTML}{B22B31}
    \definecolor{ansi-green}{HTML}{00A250}
    \definecolor{ansi-green-intense}{HTML}{007427}
    \definecolor{ansi-yellow}{HTML}{DDB62B}
    \definecolor{ansi-yellow-intense}{HTML}{B27D12}
    \definecolor{ansi-blue}{HTML}{208FFB}
    \definecolor{ansi-blue-intense}{HTML}{0065CA}
    \definecolor{ansi-magenta}{HTML}{D160C4}
    \definecolor{ansi-magenta-intense}{HTML}{A03196}
    \definecolor{ansi-cyan}{HTML}{60C6C8}
    \definecolor{ansi-cyan-intense}{HTML}{258F8F}
    \definecolor{ansi-white}{HTML}{C5C1B4}
    \definecolor{ansi-white-intense}{HTML}{A1A6B2}
    \definecolor{ansi-default-inverse-fg}{HTML}{FFFFFF}
    \definecolor{ansi-default-inverse-bg}{HTML}{000000}

    % commands and environments needed by pandoc snippets
    % extracted from the output of `pandoc -s`
    \providecommand{\tightlist}{%
      \setlength{\itemsep}{0pt}\setlength{\parskip}{0pt}}
    \DefineVerbatimEnvironment{Highlighting}{Verbatim}{commandchars=\\\{\}}
    % Add ',fontsize=\small' for more characters per line
    \newenvironment{Shaded}{}{}
    \newcommand{\KeywordTok}[1]{\textcolor[rgb]{0.00,0.44,0.13}{\textbf{{#1}}}}
    \newcommand{\DataTypeTok}[1]{\textcolor[rgb]{0.56,0.13,0.00}{{#1}}}
    \newcommand{\DecValTok}[1]{\textcolor[rgb]{0.25,0.63,0.44}{{#1}}}
    \newcommand{\BaseNTok}[1]{\textcolor[rgb]{0.25,0.63,0.44}{{#1}}}
    \newcommand{\FloatTok}[1]{\textcolor[rgb]{0.25,0.63,0.44}{{#1}}}
    \newcommand{\CharTok}[1]{\textcolor[rgb]{0.25,0.44,0.63}{{#1}}}
    \newcommand{\StringTok}[1]{\textcolor[rgb]{0.25,0.44,0.63}{{#1}}}
    \newcommand{\CommentTok}[1]{\textcolor[rgb]{0.38,0.63,0.69}{\textit{{#1}}}}
    \newcommand{\OtherTok}[1]{\textcolor[rgb]{0.00,0.44,0.13}{{#1}}}
    \newcommand{\AlertTok}[1]{\textcolor[rgb]{1.00,0.00,0.00}{\textbf{{#1}}}}
    \newcommand{\FunctionTok}[1]{\textcolor[rgb]{0.02,0.16,0.49}{{#1}}}
    \newcommand{\RegionMarkerTok}[1]{{#1}}
    \newcommand{\ErrorTok}[1]{\textcolor[rgb]{1.00,0.00,0.00}{\textbf{{#1}}}}
    \newcommand{\NormalTok}[1]{{#1}}
    
    % Additional commands for more recent versions of Pandoc
    \newcommand{\ConstantTok}[1]{\textcolor[rgb]{0.53,0.00,0.00}{{#1}}}
    \newcommand{\SpecialCharTok}[1]{\textcolor[rgb]{0.25,0.44,0.63}{{#1}}}
    \newcommand{\VerbatimStringTok}[1]{\textcolor[rgb]{0.25,0.44,0.63}{{#1}}}
    \newcommand{\SpecialStringTok}[1]{\textcolor[rgb]{0.73,0.40,0.53}{{#1}}}
    \newcommand{\ImportTok}[1]{{#1}}
    \newcommand{\DocumentationTok}[1]{\textcolor[rgb]{0.73,0.13,0.13}{\textit{{#1}}}}
    \newcommand{\AnnotationTok}[1]{\textcolor[rgb]{0.38,0.63,0.69}{\textbf{\textit{{#1}}}}}
    \newcommand{\CommentVarTok}[1]{\textcolor[rgb]{0.38,0.63,0.69}{\textbf{\textit{{#1}}}}}
    \newcommand{\VariableTok}[1]{\textcolor[rgb]{0.10,0.09,0.49}{{#1}}}
    \newcommand{\ControlFlowTok}[1]{\textcolor[rgb]{0.00,0.44,0.13}{\textbf{{#1}}}}
    \newcommand{\OperatorTok}[1]{\textcolor[rgb]{0.40,0.40,0.40}{{#1}}}
    \newcommand{\BuiltInTok}[1]{{#1}}
    \newcommand{\ExtensionTok}[1]{{#1}}
    \newcommand{\PreprocessorTok}[1]{\textcolor[rgb]{0.74,0.48,0.00}{{#1}}}
    \newcommand{\AttributeTok}[1]{\textcolor[rgb]{0.49,0.56,0.16}{{#1}}}
    \newcommand{\InformationTok}[1]{\textcolor[rgb]{0.38,0.63,0.69}{\textbf{\textit{{#1}}}}}
    \newcommand{\WarningTok}[1]{\textcolor[rgb]{0.38,0.63,0.69}{\textbf{\textit{{#1}}}}}
    
    
    % Define a nice break command that doesn't care if a line doesn't already
    % exist.
    \def\br{\hspace*{\fill} \\* }
    % Math Jax compatibility definitions
    \def\gt{>}
    \def\lt{<}
    \let\Oldtex\TeX
    \let\Oldlatex\LaTeX
    \renewcommand{\TeX}{\textrm{\Oldtex}}
    \renewcommand{\LaTeX}{\textrm{\Oldlatex}}
    % Document parameters
    % Document title
    \title{homework1-skel}
    
    
    
    
    
% Pygments definitions
\makeatletter
\def\PY@reset{\let\PY@it=\relax \let\PY@bf=\relax%
    \let\PY@ul=\relax \let\PY@tc=\relax%
    \let\PY@bc=\relax \let\PY@ff=\relax}
\def\PY@tok#1{\csname PY@tok@#1\endcsname}
\def\PY@toks#1+{\ifx\relax#1\empty\else%
    \PY@tok{#1}\expandafter\PY@toks\fi}
\def\PY@do#1{\PY@bc{\PY@tc{\PY@ul{%
    \PY@it{\PY@bf{\PY@ff{#1}}}}}}}
\def\PY#1#2{\PY@reset\PY@toks#1+\relax+\PY@do{#2}}

\expandafter\def\csname PY@tok@w\endcsname{\def\PY@tc##1{\textcolor[rgb]{0.73,0.73,0.73}{##1}}}
\expandafter\def\csname PY@tok@c\endcsname{\let\PY@it=\textit\def\PY@tc##1{\textcolor[rgb]{0.25,0.50,0.50}{##1}}}
\expandafter\def\csname PY@tok@cp\endcsname{\def\PY@tc##1{\textcolor[rgb]{0.74,0.48,0.00}{##1}}}
\expandafter\def\csname PY@tok@k\endcsname{\let\PY@bf=\textbf\def\PY@tc##1{\textcolor[rgb]{0.00,0.50,0.00}{##1}}}
\expandafter\def\csname PY@tok@kp\endcsname{\def\PY@tc##1{\textcolor[rgb]{0.00,0.50,0.00}{##1}}}
\expandafter\def\csname PY@tok@kt\endcsname{\def\PY@tc##1{\textcolor[rgb]{0.69,0.00,0.25}{##1}}}
\expandafter\def\csname PY@tok@o\endcsname{\def\PY@tc##1{\textcolor[rgb]{0.40,0.40,0.40}{##1}}}
\expandafter\def\csname PY@tok@ow\endcsname{\let\PY@bf=\textbf\def\PY@tc##1{\textcolor[rgb]{0.67,0.13,1.00}{##1}}}
\expandafter\def\csname PY@tok@nb\endcsname{\def\PY@tc##1{\textcolor[rgb]{0.00,0.50,0.00}{##1}}}
\expandafter\def\csname PY@tok@nf\endcsname{\def\PY@tc##1{\textcolor[rgb]{0.00,0.00,1.00}{##1}}}
\expandafter\def\csname PY@tok@nc\endcsname{\let\PY@bf=\textbf\def\PY@tc##1{\textcolor[rgb]{0.00,0.00,1.00}{##1}}}
\expandafter\def\csname PY@tok@nn\endcsname{\let\PY@bf=\textbf\def\PY@tc##1{\textcolor[rgb]{0.00,0.00,1.00}{##1}}}
\expandafter\def\csname PY@tok@ne\endcsname{\let\PY@bf=\textbf\def\PY@tc##1{\textcolor[rgb]{0.82,0.25,0.23}{##1}}}
\expandafter\def\csname PY@tok@nv\endcsname{\def\PY@tc##1{\textcolor[rgb]{0.10,0.09,0.49}{##1}}}
\expandafter\def\csname PY@tok@no\endcsname{\def\PY@tc##1{\textcolor[rgb]{0.53,0.00,0.00}{##1}}}
\expandafter\def\csname PY@tok@nl\endcsname{\def\PY@tc##1{\textcolor[rgb]{0.63,0.63,0.00}{##1}}}
\expandafter\def\csname PY@tok@ni\endcsname{\let\PY@bf=\textbf\def\PY@tc##1{\textcolor[rgb]{0.60,0.60,0.60}{##1}}}
\expandafter\def\csname PY@tok@na\endcsname{\def\PY@tc##1{\textcolor[rgb]{0.49,0.56,0.16}{##1}}}
\expandafter\def\csname PY@tok@nt\endcsname{\let\PY@bf=\textbf\def\PY@tc##1{\textcolor[rgb]{0.00,0.50,0.00}{##1}}}
\expandafter\def\csname PY@tok@nd\endcsname{\def\PY@tc##1{\textcolor[rgb]{0.67,0.13,1.00}{##1}}}
\expandafter\def\csname PY@tok@s\endcsname{\def\PY@tc##1{\textcolor[rgb]{0.73,0.13,0.13}{##1}}}
\expandafter\def\csname PY@tok@sd\endcsname{\let\PY@it=\textit\def\PY@tc##1{\textcolor[rgb]{0.73,0.13,0.13}{##1}}}
\expandafter\def\csname PY@tok@si\endcsname{\let\PY@bf=\textbf\def\PY@tc##1{\textcolor[rgb]{0.73,0.40,0.53}{##1}}}
\expandafter\def\csname PY@tok@se\endcsname{\let\PY@bf=\textbf\def\PY@tc##1{\textcolor[rgb]{0.73,0.40,0.13}{##1}}}
\expandafter\def\csname PY@tok@sr\endcsname{\def\PY@tc##1{\textcolor[rgb]{0.73,0.40,0.53}{##1}}}
\expandafter\def\csname PY@tok@ss\endcsname{\def\PY@tc##1{\textcolor[rgb]{0.10,0.09,0.49}{##1}}}
\expandafter\def\csname PY@tok@sx\endcsname{\def\PY@tc##1{\textcolor[rgb]{0.00,0.50,0.00}{##1}}}
\expandafter\def\csname PY@tok@m\endcsname{\def\PY@tc##1{\textcolor[rgb]{0.40,0.40,0.40}{##1}}}
\expandafter\def\csname PY@tok@gh\endcsname{\let\PY@bf=\textbf\def\PY@tc##1{\textcolor[rgb]{0.00,0.00,0.50}{##1}}}
\expandafter\def\csname PY@tok@gu\endcsname{\let\PY@bf=\textbf\def\PY@tc##1{\textcolor[rgb]{0.50,0.00,0.50}{##1}}}
\expandafter\def\csname PY@tok@gd\endcsname{\def\PY@tc##1{\textcolor[rgb]{0.63,0.00,0.00}{##1}}}
\expandafter\def\csname PY@tok@gi\endcsname{\def\PY@tc##1{\textcolor[rgb]{0.00,0.63,0.00}{##1}}}
\expandafter\def\csname PY@tok@gr\endcsname{\def\PY@tc##1{\textcolor[rgb]{1.00,0.00,0.00}{##1}}}
\expandafter\def\csname PY@tok@ge\endcsname{\let\PY@it=\textit}
\expandafter\def\csname PY@tok@gs\endcsname{\let\PY@bf=\textbf}
\expandafter\def\csname PY@tok@gp\endcsname{\let\PY@bf=\textbf\def\PY@tc##1{\textcolor[rgb]{0.00,0.00,0.50}{##1}}}
\expandafter\def\csname PY@tok@go\endcsname{\def\PY@tc##1{\textcolor[rgb]{0.53,0.53,0.53}{##1}}}
\expandafter\def\csname PY@tok@gt\endcsname{\def\PY@tc##1{\textcolor[rgb]{0.00,0.27,0.87}{##1}}}
\expandafter\def\csname PY@tok@err\endcsname{\def\PY@bc##1{\setlength{\fboxsep}{0pt}\fcolorbox[rgb]{1.00,0.00,0.00}{1,1,1}{\strut ##1}}}
\expandafter\def\csname PY@tok@kc\endcsname{\let\PY@bf=\textbf\def\PY@tc##1{\textcolor[rgb]{0.00,0.50,0.00}{##1}}}
\expandafter\def\csname PY@tok@kd\endcsname{\let\PY@bf=\textbf\def\PY@tc##1{\textcolor[rgb]{0.00,0.50,0.00}{##1}}}
\expandafter\def\csname PY@tok@kn\endcsname{\let\PY@bf=\textbf\def\PY@tc##1{\textcolor[rgb]{0.00,0.50,0.00}{##1}}}
\expandafter\def\csname PY@tok@kr\endcsname{\let\PY@bf=\textbf\def\PY@tc##1{\textcolor[rgb]{0.00,0.50,0.00}{##1}}}
\expandafter\def\csname PY@tok@bp\endcsname{\def\PY@tc##1{\textcolor[rgb]{0.00,0.50,0.00}{##1}}}
\expandafter\def\csname PY@tok@fm\endcsname{\def\PY@tc##1{\textcolor[rgb]{0.00,0.00,1.00}{##1}}}
\expandafter\def\csname PY@tok@vc\endcsname{\def\PY@tc##1{\textcolor[rgb]{0.10,0.09,0.49}{##1}}}
\expandafter\def\csname PY@tok@vg\endcsname{\def\PY@tc##1{\textcolor[rgb]{0.10,0.09,0.49}{##1}}}
\expandafter\def\csname PY@tok@vi\endcsname{\def\PY@tc##1{\textcolor[rgb]{0.10,0.09,0.49}{##1}}}
\expandafter\def\csname PY@tok@vm\endcsname{\def\PY@tc##1{\textcolor[rgb]{0.10,0.09,0.49}{##1}}}
\expandafter\def\csname PY@tok@sa\endcsname{\def\PY@tc##1{\textcolor[rgb]{0.73,0.13,0.13}{##1}}}
\expandafter\def\csname PY@tok@sb\endcsname{\def\PY@tc##1{\textcolor[rgb]{0.73,0.13,0.13}{##1}}}
\expandafter\def\csname PY@tok@sc\endcsname{\def\PY@tc##1{\textcolor[rgb]{0.73,0.13,0.13}{##1}}}
\expandafter\def\csname PY@tok@dl\endcsname{\def\PY@tc##1{\textcolor[rgb]{0.73,0.13,0.13}{##1}}}
\expandafter\def\csname PY@tok@s2\endcsname{\def\PY@tc##1{\textcolor[rgb]{0.73,0.13,0.13}{##1}}}
\expandafter\def\csname PY@tok@sh\endcsname{\def\PY@tc##1{\textcolor[rgb]{0.73,0.13,0.13}{##1}}}
\expandafter\def\csname PY@tok@s1\endcsname{\def\PY@tc##1{\textcolor[rgb]{0.73,0.13,0.13}{##1}}}
\expandafter\def\csname PY@tok@mb\endcsname{\def\PY@tc##1{\textcolor[rgb]{0.40,0.40,0.40}{##1}}}
\expandafter\def\csname PY@tok@mf\endcsname{\def\PY@tc##1{\textcolor[rgb]{0.40,0.40,0.40}{##1}}}
\expandafter\def\csname PY@tok@mh\endcsname{\def\PY@tc##1{\textcolor[rgb]{0.40,0.40,0.40}{##1}}}
\expandafter\def\csname PY@tok@mi\endcsname{\def\PY@tc##1{\textcolor[rgb]{0.40,0.40,0.40}{##1}}}
\expandafter\def\csname PY@tok@il\endcsname{\def\PY@tc##1{\textcolor[rgb]{0.40,0.40,0.40}{##1}}}
\expandafter\def\csname PY@tok@mo\endcsname{\def\PY@tc##1{\textcolor[rgb]{0.40,0.40,0.40}{##1}}}
\expandafter\def\csname PY@tok@ch\endcsname{\let\PY@it=\textit\def\PY@tc##1{\textcolor[rgb]{0.25,0.50,0.50}{##1}}}
\expandafter\def\csname PY@tok@cm\endcsname{\let\PY@it=\textit\def\PY@tc##1{\textcolor[rgb]{0.25,0.50,0.50}{##1}}}
\expandafter\def\csname PY@tok@cpf\endcsname{\let\PY@it=\textit\def\PY@tc##1{\textcolor[rgb]{0.25,0.50,0.50}{##1}}}
\expandafter\def\csname PY@tok@c1\endcsname{\let\PY@it=\textit\def\PY@tc##1{\textcolor[rgb]{0.25,0.50,0.50}{##1}}}
\expandafter\def\csname PY@tok@cs\endcsname{\let\PY@it=\textit\def\PY@tc##1{\textcolor[rgb]{0.25,0.50,0.50}{##1}}}

\def\PYZbs{\char`\\}
\def\PYZus{\char`\_}
\def\PYZob{\char`\{}
\def\PYZcb{\char`\}}
\def\PYZca{\char`\^}
\def\PYZam{\char`\&}
\def\PYZlt{\char`\<}
\def\PYZgt{\char`\>}
\def\PYZsh{\char`\#}
\def\PYZpc{\char`\%}
\def\PYZdl{\char`\$}
\def\PYZhy{\char`\-}
\def\PYZsq{\char`\'}
\def\PYZdq{\char`\"}
\def\PYZti{\char`\~}
% for compatibility with earlier versions
\def\PYZat{@}
\def\PYZlb{[}
\def\PYZrb{]}
\makeatother


    % For linebreaks inside Verbatim environment from package fancyvrb. 
    \makeatletter
        \newbox\Wrappedcontinuationbox 
        \newbox\Wrappedvisiblespacebox 
        \newcommand*\Wrappedvisiblespace {\textcolor{red}{\textvisiblespace}} 
        \newcommand*\Wrappedcontinuationsymbol {\textcolor{red}{\llap{\tiny$\m@th\hookrightarrow$}}} 
        \newcommand*\Wrappedcontinuationindent {3ex } 
        \newcommand*\Wrappedafterbreak {\kern\Wrappedcontinuationindent\copy\Wrappedcontinuationbox} 
        % Take advantage of the already applied Pygments mark-up to insert 
        % potential linebreaks for TeX processing. 
        %        {, <, #, %, $, ' and ": go to next line. 
        %        _, }, ^, &, >, - and ~: stay at end of broken line. 
        % Use of \textquotesingle for straight quote. 
        \newcommand*\Wrappedbreaksatspecials {% 
            \def\PYGZus{\discretionary{\char`\_}{\Wrappedafterbreak}{\char`\_}}% 
            \def\PYGZob{\discretionary{}{\Wrappedafterbreak\char`\{}{\char`\{}}% 
            \def\PYGZcb{\discretionary{\char`\}}{\Wrappedafterbreak}{\char`\}}}% 
            \def\PYGZca{\discretionary{\char`\^}{\Wrappedafterbreak}{\char`\^}}% 
            \def\PYGZam{\discretionary{\char`\&}{\Wrappedafterbreak}{\char`\&}}% 
            \def\PYGZlt{\discretionary{}{\Wrappedafterbreak\char`\<}{\char`\<}}% 
            \def\PYGZgt{\discretionary{\char`\>}{\Wrappedafterbreak}{\char`\>}}% 
            \def\PYGZsh{\discretionary{}{\Wrappedafterbreak\char`\#}{\char`\#}}% 
            \def\PYGZpc{\discretionary{}{\Wrappedafterbreak\char`\%}{\char`\%}}% 
            \def\PYGZdl{\discretionary{}{\Wrappedafterbreak\char`\$}{\char`\$}}% 
            \def\PYGZhy{\discretionary{\char`\-}{\Wrappedafterbreak}{\char`\-}}% 
            \def\PYGZsq{\discretionary{}{\Wrappedafterbreak\textquotesingle}{\textquotesingle}}% 
            \def\PYGZdq{\discretionary{}{\Wrappedafterbreak\char`\"}{\char`\"}}% 
            \def\PYGZti{\discretionary{\char`\~}{\Wrappedafterbreak}{\char`\~}}% 
        } 
        % Some characters . , ; ? ! / are not pygmentized. 
        % This macro makes them "active" and they will insert potential linebreaks 
        \newcommand*\Wrappedbreaksatpunct {% 
            \lccode`\~`\.\lowercase{\def~}{\discretionary{\hbox{\char`\.}}{\Wrappedafterbreak}{\hbox{\char`\.}}}% 
            \lccode`\~`\,\lowercase{\def~}{\discretionary{\hbox{\char`\,}}{\Wrappedafterbreak}{\hbox{\char`\,}}}% 
            \lccode`\~`\;\lowercase{\def~}{\discretionary{\hbox{\char`\;}}{\Wrappedafterbreak}{\hbox{\char`\;}}}% 
            \lccode`\~`\:\lowercase{\def~}{\discretionary{\hbox{\char`\:}}{\Wrappedafterbreak}{\hbox{\char`\:}}}% 
            \lccode`\~`\?\lowercase{\def~}{\discretionary{\hbox{\char`\?}}{\Wrappedafterbreak}{\hbox{\char`\?}}}% 
            \lccode`\~`\!\lowercase{\def~}{\discretionary{\hbox{\char`\!}}{\Wrappedafterbreak}{\hbox{\char`\!}}}% 
            \lccode`\~`\/\lowercase{\def~}{\discretionary{\hbox{\char`\/}}{\Wrappedafterbreak}{\hbox{\char`\/}}}% 
            \catcode`\.\active
            \catcode`\,\active 
            \catcode`\;\active
            \catcode`\:\active
            \catcode`\?\active
            \catcode`\!\active
            \catcode`\/\active 
            \lccode`\~`\~ 	
        }
    \makeatother

    \let\OriginalVerbatim=\Verbatim
    \makeatletter
    \renewcommand{\Verbatim}[1][1]{%
        %\parskip\z@skip
        \sbox\Wrappedcontinuationbox {\Wrappedcontinuationsymbol}%
        \sbox\Wrappedvisiblespacebox {\FV@SetupFont\Wrappedvisiblespace}%
        \def\FancyVerbFormatLine ##1{\hsize\linewidth
            \vtop{\raggedright\hyphenpenalty\z@\exhyphenpenalty\z@
                \doublehyphendemerits\z@\finalhyphendemerits\z@
                \strut ##1\strut}%
        }%
        % If the linebreak is at a space, the latter will be displayed as visible
        % space at end of first line, and a continuation symbol starts next line.
        % Stretch/shrink are however usually zero for typewriter font.
        \def\FV@Space {%
            \nobreak\hskip\z@ plus\fontdimen3\font minus\fontdimen4\font
            \discretionary{\copy\Wrappedvisiblespacebox}{\Wrappedafterbreak}
            {\kern\fontdimen2\font}%
        }%
        
        % Allow breaks at special characters using \PYG... macros.
        \Wrappedbreaksatspecials
        % Breaks at punctuation characters . , ; ? ! and / need catcode=\active 	
        \OriginalVerbatim[#1,codes*=\Wrappedbreaksatpunct]%
    }
    \makeatother

    % Exact colors from NB
    \definecolor{incolor}{HTML}{303F9F}
    \definecolor{outcolor}{HTML}{D84315}
    \definecolor{cellborder}{HTML}{CFCFCF}
    \definecolor{cellbackground}{HTML}{F7F7F7}
    
    % prompt
    \newcommand{\prompt}[4]{
        \llap{{\color{#2}[#3]: #4}}\vspace{-1.25em}
    }
    

    
    % Prevent overflowing lines due to hard-to-break entities
    \sloppy 
    % Setup hyperref package
    \hypersetup{
      breaklinks=true,  % so long urls are correctly broken across lines
      colorlinks=true,
      urlcolor=urlcolor,
      linkcolor=linkcolor,
      citecolor=citecolor,
      }
    % Slightly bigger margins than the latex defaults
    
    \geometry{verbose,tmargin=1in,bmargin=1in,lmargin=1in,rmargin=1in}
    
    

    \begin{document}
    
    
    \maketitle
    
    

    
    NAME: Nigel Mansell

\hypertarget{homework-1}{%
\section{Homework 1}\label{homework-1}}

\hypertarget{objectives}{%
\subsubsection{Objectives}\label{objectives}}

\begin{itemize}
\tightlist
\item
  Basic numpy operations to access data
\item
  Basic plotting of subsets of data
\item
  Simple descriptive statistics
\item
  Do not save work within the mlp\_2020 folder

  \begin{itemize}
  \tightlist
  \item
    create a folder in your home directory for assignments, and copy the
    templates there
  \end{itemize}
\end{itemize}

\hypertarget{general-references}{%
\subsubsection{General References}\label{general-references}}

\begin{itemize}
\tightlist
\item
  \href{https://scikit-learn.org/stable/modules/generated/sklearn.datasets.load_iris.html}{Sci-kit
  Learn Iris Dataset}
\item
  \href{https://docs.scipy.org/doc/numpy/reference/index.html}{Numpy
  Reference}
\item
  \href{https://matplotlib.org/3.1.1/api/pyplot_summary.html}{Summary of
  matplotlib}

  \begin{itemize}
  \tightlist
  \item
    \href{https://matplotlib.org/3.1.1/api/_as_gen/matplotlib.pyplot.plot.html}{Plot}
  \item
    \href{https://matplotlib.org/3.1.1/api/_as_gen/matplotlib.pyplot.boxplot.html}{Boxplots}
  \item
    \href{https://matplotlib.org/3.1.1/api/_as_gen/matplotlib.pyplot.hist.html\#matplotlib.pyplot.hist}{Histograms}
  \item
    \href{https://matplotlib.org/3.1.1/api/_as_gen/matplotlib.pyplot.scatter.html\#matplotlib.pyplot.scatter}{Scatter
    plots}
  \item
    \href{https://matplotlib.org/3.1.1/api/_as_gen/matplotlib.pyplot.imshow.html}{Colormap
    Plots}
  \end{itemize}
\end{itemize}

\hypertarget{hand-in-procedure}{%
\subsubsection{Hand-In Procedure}\label{hand-in-procedure}}

\begin{itemize}
\tightlist
\item
  Execute all cells so they are showing correct results
\item
  Notebook:

  \begin{itemize}
  \tightlist
  \item
    Submit this file (.ipynb) to the Canvas HW1 dropbox
  \end{itemize}
\item
  PDF:

  \begin{itemize}
  \tightlist
  \item
    File/Export Notebook As/PDF -\textgreater{} Produces a copy of the
    notebook in PDF format
  \item
    Submit the PDF file to the Gradescope HW1 dropbox
  \end{itemize}
\end{itemize}

    \begin{tcolorbox}[breakable, size=fbox, boxrule=1pt, pad at break*=1mm,colback=cellbackground, colframe=cellborder]
\prompt{In}{incolor}{1}{\hspace{4pt}}
\begin{Verbatim}[commandchars=\\\{\}]
\PY{k+kn}{import} \PY{n+nn}{numpy} \PY{k}{as} \PY{n+nn}{np}
\PY{k+kn}{import} \PY{n+nn}{pandas} \PY{k}{as} \PY{n+nn}{pd}
\PY{k+kn}{import} \PY{n+nn}{matplotlib}\PY{n+nn}{.}\PY{n+nn}{pyplot} \PY{k}{as} \PY{n+nn}{plt}
\PY{k+kn}{from} \PY{n+nn}{IPython} \PY{k+kn}{import} \PY{n}{get\PYZus{}ipython}
\PY{k+kn}{from} \PY{n+nn}{sklearn}\PY{n+nn}{.}\PY{n+nn}{datasets} \PY{k+kn}{import} \PY{n}{load\PYZus{}iris}
\end{Verbatim}
\end{tcolorbox}

    \hypertarget{load-iris-data-set}{%
\section{LOAD IRIS DATA SET}\label{load-iris-data-set}}

    \begin{tcolorbox}[breakable, size=fbox, boxrule=1pt, pad at break*=1mm,colback=cellbackground, colframe=cellborder]
\prompt{In}{incolor}{2}{\hspace{4pt}}
\begin{Verbatim}[commandchars=\\\{\}]
\PY{l+s+sd}{\PYZdq{}\PYZdq{}\PYZdq{}}
\PY{l+s+sd}{Load the dataset into the iris\PYZus{}dataset variable, by calling the }
\PY{l+s+sd}{load\PYZus{}iris() function imported from sklearn.datasets.}
\PY{l+s+sd}{Then display the iris\PYZus{}dataset object\PYZsq{}s list of keys. iris\PYZus{}dataset}
\PY{l+s+sd}{is a dictionary object.}
\PY{l+s+sd}{\PYZdq{}\PYZdq{}\PYZdq{}}

\PY{l+s+s2}{\PYZdq{}}\PY{l+s+s2}{laoding the data and printing the keys}\PY{l+s+s2}{\PYZdq{}}
\PY{n}{iris\PYZus{}dataset} \PY{o}{=} \PY{n}{load\PYZus{}iris}\PY{p}{(}\PY{p}{)}
\PY{n}{iris\PYZus{}dataset}\PY{o}{.}\PY{n}{keys}\PY{p}{(}\PY{p}{)}
\end{Verbatim}
\end{tcolorbox}

            \begin{tcolorbox}[breakable, boxrule=.5pt, size=fbox, pad at break*=1mm, opacityfill=0]
\prompt{Out}{outcolor}{2}{\hspace{3.5pt}}
\begin{Verbatim}[commandchars=\\\{\}]
dict\_keys(['data', 'target', 'target\_names', 'DESCR', 'feature\_names',
'filename'])
\end{Verbatim}
\end{tcolorbox}
        
    \hypertarget{dataset-details}{%
\subsubsection{Dataset Details}\label{dataset-details}}

The \texttt{iris\_dataset} variable is a dictionary with multiple
fields: * \texttt{data} : m by n numpy array of the n observed feature
values, for each of the m samples\\
* \texttt{target} : m by 1 numpy array of samples' classification as
iris-setosa (i.e.~0), iris-versicolour (i.e.~1), or iris-virginica
(i.e.~2) * \texttt{target\_names} : 3 by 1 numpy array of the possible
iris classifications\\
* \texttt{DESCR} : string containing a detailed description of the
dataset\\
* \texttt{feature\_names} : n by 1 numpy array of the names of the
feature variables\\
* \texttt{filename} : string containing the absolute path to where the
file containing all the data information is located on the local system

    \begin{tcolorbox}[breakable, size=fbox, boxrule=1pt, pad at break*=1mm,colback=cellbackground, colframe=cellborder]
\prompt{In}{incolor}{3}{\hspace{4pt}}
\begin{Verbatim}[commandchars=\\\{\}]
\PY{l+s+sd}{\PYZdq{}\PYZdq{}\PYZdq{}}
\PY{l+s+sd}{Print out the description of the data, by accessing the }
\PY{l+s+sd}{\PYZsq{}DESCR\PYZsq{} field of the iris data set}
\PY{l+s+sd}{\PYZdq{}\PYZdq{}\PYZdq{}}

\PY{l+s+s2}{\PYZdq{}}\PY{l+s+s2}{Printing the discription from its key}\PY{l+s+s2}{\PYZdq{}}
\PY{n+nb}{print}\PY{p}{(}\PY{n}{iris\PYZus{}dataset}\PY{p}{[}\PY{l+s+s1}{\PYZsq{}}\PY{l+s+s1}{DESCR}\PY{l+s+s1}{\PYZsq{}}\PY{p}{]}\PY{p}{)}
\end{Verbatim}
\end{tcolorbox}

    \begin{Verbatim}[commandchars=\\\{\}]
.. \_iris\_dataset:

Iris plants dataset
--------------------

**Data Set Characteristics:**

    :Number of Instances: 150 (50 in each of three classes)
    :Number of Attributes: 4 numeric, predictive attributes and the class
    :Attribute Information:
        - sepal length in cm
        - sepal width in cm
        - petal length in cm
        - petal width in cm
        - class:
                - Iris-Setosa
                - Iris-Versicolour
                - Iris-Virginica

    :Summary Statistics:

    ============== ==== ==== ======= ===== ====================
                    Min  Max   Mean    SD   Class Correlation
    ============== ==== ==== ======= ===== ====================
    sepal length:   4.3  7.9   5.84   0.83    0.7826
    sepal width:    2.0  4.4   3.05   0.43   -0.4194
    petal length:   1.0  6.9   3.76   1.76    0.9490  (high!)
    petal width:    0.1  2.5   1.20   0.76    0.9565  (high!)
    ============== ==== ==== ======= ===== ====================

    :Missing Attribute Values: None
    :Class Distribution: 33.3\% for each of 3 classes.
    :Creator: R.A. Fisher
    :Donor: Michael Marshall (MARSHALL\%PLU@io.arc.nasa.gov)
    :Date: July, 1988

The famous Iris database, first used by Sir R.A. Fisher. The dataset is taken
from Fisher's paper. Note that it's the same as in R, but not as in the UCI
Machine Learning Repository, which has two wrong data points.

This is perhaps the best known database to be found in the
pattern recognition literature.  Fisher's paper is a classic in the field and
is referenced frequently to this day.  (See Duda \& Hart, for example.)  The
data set contains 3 classes of 50 instances each, where each class refers to a
type of iris plant.  One class is linearly separable from the other 2; the
latter are NOT linearly separable from each other.

.. topic:: References

   - Fisher, R.A. "The use of multiple measurements in taxonomic problems"
     Annual Eugenics, 7, Part II, 179-188 (1936); also in "Contributions to
     Mathematical Statistics" (John Wiley, NY, 1950).
   - Duda, R.O., \& Hart, P.E. (1973) Pattern Classification and Scene Analysis.
     (Q327.D83) John Wiley \& Sons.  ISBN 0-471-22361-1.  See page 218.
   - Dasarathy, B.V. (1980) "Nosing Around the Neighborhood: A New System
     Structure and Classification Rule for Recognition in Partially Exposed
     Environments".  IEEE Transactions on Pattern Analysis and Machine
     Intelligence, Vol. PAMI-2, No. 1, 67-71.
   - Gates, G.W. (1972) "The Reduced Nearest Neighbor Rule".  IEEE Transactions
     on Information Theory, May 1972, 431-433.
   - See also: 1988 MLC Proceedings, 54-64.  Cheeseman et al"s AUTOCLASS II
     conceptual clustering system finds 3 classes in the data.
   - Many, many more {\ldots}
\end{Verbatim}

    \hypertarget{setup-useful-variables}{%
\subsection{SETUP USEFUL VARIABLES}\label{setup-useful-variables}}

    \begin{tcolorbox}[breakable, size=fbox, boxrule=1pt, pad at break*=1mm,colback=cellbackground, colframe=cellborder]
\prompt{In}{incolor}{4}{\hspace{4pt}}
\begin{Verbatim}[commandchars=\\\{\}]
\PY{l+s+sd}{\PYZdq{}\PYZdq{}\PYZdq{}}
\PY{l+s+sd}{Store the names of the features and the names }
\PY{l+s+sd}{of the target classes, into the variables}
\PY{l+s+sd}{feature\PYZus{}names and target\PYZus{}names respectively.}
\PY{l+s+sd}{\PYZdq{}\PYZdq{}\PYZdq{}}

\PY{l+s+s2}{\PYZdq{}}\PY{l+s+s2}{seting feature and target names to their associated keys}\PY{l+s+s2}{\PYZdq{}}
\PY{n}{feature\PYZus{}names} \PY{o}{=}  \PY{n}{iris\PYZus{}dataset}\PY{p}{[}\PY{l+s+s1}{\PYZsq{}}\PY{l+s+s1}{feature\PYZus{}names}\PY{l+s+s1}{\PYZsq{}}\PY{p}{]}
\PY{n}{target\PYZus{}names} \PY{o}{=} \PY{n}{iris\PYZus{}dataset}\PY{p}{[}\PY{l+s+s1}{\PYZsq{}}\PY{l+s+s1}{target\PYZus{}names}\PY{l+s+s1}{\PYZsq{}}\PY{p}{]}

\PY{l+s+sd}{\PYZdq{}\PYZdq{}\PYZdq{}}
\PY{l+s+sd}{Print the list of feature names and target names}
\PY{l+s+sd}{\PYZdq{}\PYZdq{}\PYZdq{}}

\PY{l+s+s2}{\PYZdq{}}\PY{l+s+s2}{printing feature and target names}\PY{l+s+s2}{\PYZdq{}}
\PY{n+nb}{print}\PY{p}{(}\PY{n}{feature\PYZus{}names}\PY{p}{)}
\PY{n+nb}{print}\PY{p}{(}\PY{n}{target\PYZus{}names}\PY{p}{)}
\end{Verbatim}
\end{tcolorbox}

    \begin{Verbatim}[commandchars=\\\{\}]
['sepal length (cm)', 'sepal width (cm)', 'petal length (cm)', 'petal width
(cm)']
['setosa' 'versicolor' 'virginica']
\end{Verbatim}

    \begin{tcolorbox}[breakable, size=fbox, boxrule=1pt, pad at break*=1mm,colback=cellbackground, colframe=cellborder]
\prompt{In}{incolor}{5}{\hspace{4pt}}
\begin{Verbatim}[commandchars=\\\{\}]
\PY{l+s+sd}{\PYZdq{}\PYZdq{}\PYZdq{} }
\PY{l+s+sd}{Create variables for the feature and target data }
\PY{l+s+sd}{The X variable is a numpy array containing the data measured }
\PY{l+s+sd}{for each feature for each sample. Each column of X is a }
\PY{l+s+sd}{different feature for all the samples. Each row of X is a }
\PY{l+s+sd}{different sample with all its features.}
\PY{l+s+sd}{The y variable is a numpy array containing the classification }
\PY{l+s+sd}{for each sample. A sample iris is either setosa, versicolor, or virginica.}
\PY{l+s+sd}{\PYZdq{}\PYZdq{}\PYZdq{}} 

\PY{l+s+s2}{\PYZdq{}}\PY{l+s+s2}{setting X to the data from the key data and y to data from target}\PY{l+s+s2}{\PYZdq{}}
\PY{n}{X} \PY{o}{=} \PY{n}{iris\PYZus{}dataset}\PY{p}{[}\PY{l+s+s1}{\PYZsq{}}\PY{l+s+s1}{data}\PY{l+s+s1}{\PYZsq{}}\PY{p}{]}
\PY{n}{y} \PY{o}{=} \PY{n}{iris\PYZus{}dataset}\PY{p}{[}\PY{l+s+s1}{\PYZsq{}}\PY{l+s+s1}{target}\PY{l+s+s1}{\PYZsq{}}\PY{p}{]}

\PY{l+s+sd}{\PYZdq{}\PYZdq{}\PYZdq{}}
\PY{l+s+sd}{Print the dimensions of the X and y variables respectively}
\PY{l+s+sd}{\PYZdq{}\PYZdq{}\PYZdq{}}

\PY{l+s+s2}{\PYZdq{}}\PY{l+s+s2}{printing both x and y dimensions}\PY{l+s+s2}{\PYZdq{}}
\PY{n+nb}{print}\PY{p}{(}\PY{n}{X}\PY{o}{.}\PY{n}{ndim}\PY{p}{)}
\PY{n+nb}{print}\PY{p}{(}\PY{n}{y}\PY{o}{.}\PY{n}{ndim}\PY{p}{)}
\end{Verbatim}
\end{tcolorbox}

    \begin{Verbatim}[commandchars=\\\{\}]
2
1
\end{Verbatim}

    \begin{tcolorbox}[breakable, size=fbox, boxrule=1pt, pad at break*=1mm,colback=cellbackground, colframe=cellborder]
\prompt{In}{incolor}{6}{\hspace{4pt}}
\begin{Verbatim}[commandchars=\\\{\}]
\PY{l+s+sd}{\PYZdq{}\PYZdq{}\PYZdq{} }
\PY{l+s+sd}{Store the number of samples and the number of features, by}
\PY{l+s+sd}{accessing the values from the shape of X}
\PY{l+s+sd}{\PYZdq{}\PYZdq{}\PYZdq{}}

\PY{l+s+s2}{\PYZdq{}}\PY{l+s+s2}{setting nsamples to the row of X and nfeatures to the column of X}\PY{l+s+s2}{\PYZdq{}}
\PY{n}{nsamples}\PY{p}{,} \PY{n}{nfeatures} \PY{o}{=} \PY{n}{X}\PY{o}{.}\PY{n}{shape}

\PY{l+s+sd}{\PYZdq{}\PYZdq{}\PYZdq{}}
\PY{l+s+sd}{Print the print the number of samples and numberof features respectively}
\PY{l+s+sd}{\PYZdq{}\PYZdq{}\PYZdq{}}

\PY{l+s+s2}{\PYZdq{}}\PY{l+s+s2}{printing nsamples and nfeatures}\PY{l+s+s2}{\PYZdq{}}
\PY{n+nb}{print}\PY{p}{(}\PY{n}{nsamples}\PY{p}{)}
\PY{n+nb}{print}\PY{p}{(}\PY{n}{nfeatures}\PY{p}{)}
\end{Verbatim}
\end{tcolorbox}

    \begin{Verbatim}[commandchars=\\\{\}]
150
4
\end{Verbatim}

    \hypertarget{select-subset-of-features}{%
\subsection{SELECT SUBSET OF FEATURES}\label{select-subset-of-features}}

Not all available data is necessary or useful for making predictions and
classifying observations. There are numerous feature selection
algorithms that exist, which will be discussed in more detail later
within the cousre. For now we are going to arbitrarly select sepal
length, sepal width and petal length as our predictor variables. We will
not yet be performing any predictions in this assignment; rather this
term is used to conveniently distinuguish this subset of features from
the full set of features.

    \begin{tcolorbox}[breakable, size=fbox, boxrule=1pt, pad at break*=1mm,colback=cellbackground, colframe=cellborder]
\prompt{In}{incolor}{7}{\hspace{4pt}}
\begin{Verbatim}[commandchars=\\\{\}]
\PY{l+s+sd}{\PYZdq{}\PYZdq{}\PYZdq{} PROVIDED}
\PY{l+s+sd}{Feature Column Indices}
\PY{l+s+sd}{The values observed for each feature resides within a particular }
\PY{l+s+sd}{column of the feature matrix, X. For example, column 0 contains the }
\PY{l+s+sd}{values of the mean radius for each observation, the column at index }
\PY{l+s+sd}{3 contains the values for the mean area, and so on.}
\PY{l+s+sd}{\PYZdq{}\PYZdq{}\PYZdq{}}
\PY{n}{sepal\PYZus{}length\PYZus{}idx} \PY{o}{=} \PY{l+m+mi}{0}
\PY{n}{sepal\PYZus{}width\PYZus{}idx} \PY{o}{=} \PY{l+m+mi}{1}
\PY{n}{petal\PYZus{}length\PYZus{}idx} \PY{o}{=} \PY{l+m+mi}{2}

\PY{l+s+sd}{\PYZdq{}\PYZdq{}\PYZdq{}}
\PY{l+s+sd}{Create a list of the select subset of features}
\PY{l+s+sd}{\PYZdq{}\PYZdq{}\PYZdq{}}
\PY{n}{predictors} \PY{o}{=} \PY{p}{[}\PY{n}{sepal\PYZus{}length\PYZus{}idx}\PY{p}{,} \PY{n}{sepal\PYZus{}width\PYZus{}idx}\PY{p}{,} \PY{n}{petal\PYZus{}length\PYZus{}idx}\PY{p}{]}

\PY{l+s+sd}{\PYZdq{}\PYZdq{}\PYZdq{}}
\PY{l+s+sd}{Create a variable, storing the number of predictors}
\PY{l+s+sd}{\PYZdq{}\PYZdq{}\PYZdq{}}


\PY{l+s+s2}{\PYZdq{}}\PY{l+s+s2}{the length of predictors}\PY{l+s+s2}{\PYZdq{}}
\PY{n}{predictorSize} \PY{o}{=} \PY{n+nb}{len}\PY{p}{(}\PY{n}{predictors}\PY{p}{)}

\PY{l+s+sd}{\PYZdq{}\PYZdq{}\PYZdq{}}
\PY{l+s+sd}{Create a list of corresponding names for the selected set of features.}
\PY{l+s+sd}{This is conveniently done using list comprehension}
\PY{l+s+sd}{\PYZdq{}\PYZdq{}\PYZdq{}}

\PY{l+s+s2}{\PYZdq{}}\PY{l+s+s2}{the names of the predictors from feature\PYZus{}names array}\PY{l+s+s2}{\PYZdq{}}
\PY{n}{predictorNames} \PY{o}{=} \PY{p}{[}\PY{n}{feature\PYZus{}names}\PY{p}{[}\PY{n}{i}\PY{p}{]} \PY{k}{for} \PY{n}{i} \PY{o+ow}{in} \PY{n+nb}{range}\PY{p}{(}\PY{n}{predictorSize}\PY{p}{)}\PY{p}{]}
\PY{l+s+sd}{\PYZdq{}\PYZdq{}\PYZdq{}}
\PY{l+s+sd}{Print the list of predictor names}
\PY{l+s+sd}{\PYZdq{}\PYZdq{}\PYZdq{}}

\PY{l+s+s2}{\PYZdq{}}\PY{l+s+s2}{prints the names}\PY{l+s+s2}{\PYZdq{}}
\PY{n+nb}{print}\PY{p}{(}\PY{n}{predictorNames}\PY{p}{)}
\end{Verbatim}
\end{tcolorbox}

    \begin{Verbatim}[commandchars=\\\{\}]
['sepal length (cm)', 'sepal width (cm)', 'petal length (cm)']
\end{Verbatim}

    \hypertarget{basic-histograms-of-features}{%
\subsection{BASIC HISTOGRAMS OF
FEATURES}\label{basic-histograms-of-features}}

    \begin{tcolorbox}[breakable, size=fbox, boxrule=1pt, pad at break*=1mm,colback=cellbackground, colframe=cellborder]
\prompt{In}{incolor}{8}{\hspace{4pt}}
\begin{Verbatim}[commandchars=\\\{\}]
\PY{l+s+sd}{\PYZdq{}\PYZdq{}\PYZdq{} TODO}
\PY{l+s+sd}{HISTOGRAMS OF THE CHOSEN PREDICTOR FEATURES}
\PY{l+s+sd}{Please plot histograms in their own subplot of }
\PY{l+s+sd}{the same figure.}
\PY{l+s+sd}{\PYZdq{}\PYZdq{}\PYZdq{}}

\PY{l+s+s2}{\PYZdq{}}\PY{l+s+s2}{Turned X into a dataframe in order to iterate over columns easier}\PY{l+s+s2}{\PYZdq{}}
\PY{n}{df} \PY{o}{=} \PY{n}{pd}\PY{o}{.}\PY{n}{DataFrame}\PY{p}{(}\PY{n}{data}\PY{o}{=}\PY{n}{X}\PY{p}{,}\PY{n}{index}\PY{o}{=}\PY{k+kc}{None}\PY{p}{,}\PY{n}{columns}\PY{o}{=}\PY{n}{feature\PYZus{}names}\PY{p}{)}
\PY{n}{plt}\PY{o}{.}\PY{n}{figure}\PY{p}{(}\PY{n}{figsize}\PY{o}{=}\PY{p}{(}\PY{l+m+mi}{20}\PY{p}{,}\PY{l+m+mi}{4}\PY{p}{)}\PY{p}{)}
\PY{n}{plt}\PY{o}{.}\PY{n}{subplots\PYZus{}adjust}\PY{p}{(}\PY{n}{wspace}\PY{o}{=}\PY{o}{.}\PY{l+m+mi}{3}\PY{p}{)}
\PY{k}{for} \PY{n}{i}\PY{p}{,} \PY{n}{fidx} \PY{o+ow}{in} \PY{n+nb}{enumerate}\PY{p}{(}\PY{n}{predictors}\PY{p}{,} \PY{l+m+mi}{1}\PY{p}{)}\PY{p}{:}
    \PY{n}{plt}\PY{o}{.}\PY{n}{subplot}\PY{p}{(}\PY{l+m+mi}{1}\PY{p}{,} \PY{l+m+mi}{3}\PY{p}{,} \PY{n}{i}\PY{p}{)}
    \PY{n}{plt}\PY{o}{.}\PY{n}{title}\PY{p}{(}\PY{n}{predictorNames}\PY{p}{[}\PY{n}{fidx}\PY{p}{]}\PY{p}{)}
    \PY{n}{df}\PY{p}{[}\PY{n}{predictorNames}\PY{p}{[}\PY{n}{fidx}\PY{p}{]}\PY{p}{]}\PY{o}{.}\PY{n}{hist}\PY{p}{(}\PY{p}{)}
    \PY{c+c1}{\PYZsh{} TODO: Plot the histogram }
\end{Verbatim}
\end{tcolorbox}

    \begin{center}
    \adjustimage{max size={0.9\linewidth}{0.9\paperheight}}{output_13_0.png}
    \end{center}
    { \hspace*{\fill} \\}
    
    \begin{tcolorbox}[breakable, size=fbox, boxrule=1pt, pad at break*=1mm,colback=cellbackground, colframe=cellborder]
\prompt{In}{incolor}{9}{\hspace{4pt}}
\begin{Verbatim}[commandchars=\\\{\}]
\PY{l+s+sd}{\PYZdq{}\PYZdq{}\PYZdq{} TODO}
\PY{l+s+sd}{Create a histogram or barplot for the counts}
\PY{l+s+sd}{for each target class}
\PY{l+s+sd}{\PYZdq{}\PYZdq{}\PYZdq{}}

\PY{l+s+s2}{\PYZdq{}}\PY{l+s+s2}{creates 3 histograms for each class and displays their count, or how often they appear}\PY{l+s+s2}{\PYZdq{}}
\PY{k}{for} \PY{n}{i} \PY{o+ow}{in} \PY{n}{y}\PY{p}{:}
    \PY{n}{plt}\PY{o}{.}\PY{n}{figure}\PY{p}{(}\PY{n}{i}\PY{o}{+}\PY{l+m+mi}{1}\PY{p}{)}
    \PY{n}{plt}\PY{o}{.}\PY{n}{title}\PY{p}{(}\PY{n}{target\PYZus{}names}\PY{p}{[}\PY{n}{i}\PY{p}{]}\PY{p}{)}
    \PY{n}{plt}\PY{o}{.}\PY{n}{hist}\PY{p}{(}\PY{n}{y}\PY{p}{,} \PY{n}{bins}\PY{o}{=}\PY{n}{np}\PY{o}{.}\PY{n}{arange}\PY{p}{(}\PY{l+m+mi}{0}\PY{p}{,} \PY{l+m+mi}{51}\PY{p}{)}\PY{p}{)}
\end{Verbatim}
\end{tcolorbox}

    \begin{center}
    \adjustimage{max size={0.9\linewidth}{0.9\paperheight}}{output_14_0.png}
    \end{center}
    { \hspace*{\fill} \\}
    
    \begin{center}
    \adjustimage{max size={0.9\linewidth}{0.9\paperheight}}{output_14_1.png}
    \end{center}
    { \hspace*{\fill} \\}
    
    \begin{center}
    \adjustimage{max size={0.9\linewidth}{0.9\paperheight}}{output_14_2.png}
    \end{center}
    { \hspace*{\fill} \\}
    
    \hypertarget{basic-boxplots-of-features}{%
\subsection{BASIC BOXPLOTS OF
FEATURES}\label{basic-boxplots-of-features}}

Boxplots or box-and-whisker plots are used to obtain a perspective of
the distribution of the data. The box within the figure displays the
25th percentile (Q1), the median, and the 75th percentile (Q3) of the
data. The range between the 75th percentile value and the 25th
percentile value is the interquartile range (IQR = Q3 - Q1). The end of
bottom line is Q1 - 1.5 * IQR. The end of top line is Q3 + 1.5 * IQR.
Anything beyond the lines, the circles, are suggested outliers.\\

\textless{}\center\textgreater{}

One can use the \texttt{boxplot(data\_values,\ labels={[}name{]})} to
generate a boxplot. \texttt{data\_values} would be the set of observed
values for a paritucular feature and \texttt{labels} should be provided
as a list, with the name of the feature, in place of \texttt{name}.

    \begin{tcolorbox}[breakable, size=fbox, boxrule=1pt, pad at break*=1mm,colback=cellbackground, colframe=cellborder]
\prompt{In}{incolor}{10}{\hspace{4pt}}
\begin{Verbatim}[commandchars=\\\{\}]
\PY{l+s+sd}{\PYZdq{}\PYZdq{}\PYZdq{} TODO}
\PY{l+s+sd}{BOXPLOTS OF THE CHOSEN PREDICTOR FEATURES}
\PY{l+s+sd}{Please place the boxplots within their own }
\PY{l+s+sd}{subplot of the same figure }
\PY{l+s+sd}{\PYZdq{}\PYZdq{}\PYZdq{}}

\PY{l+s+s2}{\PYZdq{}}\PY{l+s+s2}{Creates boxplots for predictors}\PY{l+s+s2}{\PYZdq{}}
\PY{n}{plt}\PY{o}{.}\PY{n}{subplots\PYZus{}adjust}\PY{p}{(}\PY{n}{wspace}\PY{o}{=}\PY{o}{.}\PY{l+m+mi}{5}\PY{p}{)}
\PY{k}{for} \PY{n}{i}\PY{p}{,} \PY{n}{fidx} \PY{o+ow}{in} \PY{n+nb}{enumerate}\PY{p}{(}\PY{n}{predictors}\PY{p}{,} \PY{l+m+mi}{1}\PY{p}{)}\PY{p}{:}
    \PY{n}{plt}\PY{o}{.}\PY{n}{subplot}\PY{p}{(}\PY{l+m+mi}{1}\PY{p}{,} \PY{l+m+mi}{3}\PY{p}{,} \PY{n}{i}\PY{p}{)}
    \PY{n}{df}\PY{o}{.}\PY{n}{boxplot}\PY{p}{(}\PY{n}{column}\PY{o}{=}\PY{n}{predictorNames}\PY{p}{[}\PY{n}{fidx}\PY{p}{]}\PY{p}{)}
\end{Verbatim}
\end{tcolorbox}

    \begin{center}
    \adjustimage{max size={0.9\linewidth}{0.9\paperheight}}{output_16_0.png}
    \end{center}
    { \hspace*{\fill} \\}
    
    \hypertarget{descriptive-statistics}{%
\subsection{DESCRIPTIVE STATISTICS}\label{descriptive-statistics}}

    \begin{tcolorbox}[breakable, size=fbox, boxrule=1pt, pad at break*=1mm,colback=cellbackground, colframe=cellborder]
\prompt{In}{incolor}{11}{\hspace{4pt}}
\begin{Verbatim}[commandchars=\\\{\}]
\PY{c+c1}{\PYZsh{} Simply run this cell}
\PY{l+s+sd}{\PYZdq{}\PYZdq{}\PYZdq{} }
\PY{l+s+sd}{Create a separate variable of the data from the }
\PY{l+s+sd}{predictors}
\PY{l+s+sd}{\PYZdq{}\PYZdq{}\PYZdq{}}
\PY{n}{Xpreds} \PY{o}{=} \PY{n}{X}\PY{p}{[}\PY{p}{:}\PY{p}{,} \PY{n}{predictors}\PY{p}{]}

\PY{l+s+sd}{\PYZdq{}\PYZdq{}\PYZdq{}}
\PY{l+s+sd}{Check if any values are NaN (not a number)}
\PY{l+s+sd}{\PYZdq{}\PYZdq{}\PYZdq{}}
\PY{n}{np}\PY{o}{.}\PY{n}{any}\PY{p}{(}\PY{n}{np}\PY{o}{.}\PY{n}{isnan}\PY{p}{(}\PY{n}{Xpreds}\PY{p}{)}\PY{p}{)}
\end{Verbatim}
\end{tcolorbox}

            \begin{tcolorbox}[breakable, boxrule=.5pt, size=fbox, pad at break*=1mm, opacityfill=0]
\prompt{Out}{outcolor}{11}{\hspace{3.5pt}}
\begin{Verbatim}[commandchars=\\\{\}]
False
\end{Verbatim}
\end{tcolorbox}
        
    \begin{tcolorbox}[breakable, size=fbox, boxrule=1pt, pad at break*=1mm,colback=cellbackground, colframe=cellborder]
\prompt{In}{incolor}{12}{\hspace{4pt}}
\begin{Verbatim}[commandchars=\\\{\}]
\PY{l+s+sd}{\PYZdq{}\PYZdq{}\PYZdq{} TODO}
\PY{l+s+sd}{Compute the following descriptive statistics of the }
\PY{l+s+sd}{features ignoring NaN values, using numpy:}
\PY{l+s+sd}{mean, median, standard deviation, min, and max}

\PY{l+s+sd}{Make sure to compute the statistics of the columns}
\PY{l+s+sd}{of X (i.e. of each feature). You can specify this }
\PY{l+s+sd}{by setting axis=0 for each of the functions}

\PY{l+s+sd}{Compute and print the results}
\PY{l+s+sd}{\PYZdq{}\PYZdq{}\PYZdq{}}

\PY{l+s+s2}{\PYZdq{}}\PY{l+s+s2}{Prints the non nan vales of each statistic}\PY{l+s+s2}{\PYZdq{}}
\PY{n+nb}{print}\PY{p}{(}\PY{l+s+s2}{\PYZdq{}}\PY{l+s+s2}{Mean: }\PY{l+s+s2}{\PYZdq{}}\PY{p}{,} \PY{n}{np}\PY{o}{.}\PY{n}{nanmean}\PY{p}{(}\PY{n}{Xpreds}\PY{p}{,} \PY{n}{axis}\PY{o}{=}\PY{l+m+mi}{0}\PY{p}{)}\PY{p}{)}
\PY{n+nb}{print}\PY{p}{(}\PY{l+s+s2}{\PYZdq{}}\PY{l+s+s2}{Median: }\PY{l+s+s2}{\PYZdq{}}\PY{p}{,} \PY{n}{np}\PY{o}{.}\PY{n}{nanmedian}\PY{p}{(}\PY{n}{Xpreds}\PY{p}{,} \PY{n}{axis}\PY{o}{=}\PY{l+m+mi}{0}\PY{p}{)}\PY{p}{)}
\PY{n+nb}{print}\PY{p}{(}\PY{l+s+s2}{\PYZdq{}}\PY{l+s+s2}{Standard deviation: }\PY{l+s+s2}{\PYZdq{}}\PY{p}{,} \PY{n}{np}\PY{o}{.}\PY{n}{nanstd}\PY{p}{(}\PY{n}{Xpreds}\PY{p}{,} \PY{n}{axis}\PY{o}{=}\PY{l+m+mi}{0}\PY{p}{)}\PY{p}{)}
\PY{n+nb}{print}\PY{p}{(}\PY{l+s+s2}{\PYZdq{}}\PY{l+s+s2}{Min: }\PY{l+s+s2}{\PYZdq{}}\PY{p}{,} \PY{n}{np}\PY{o}{.}\PY{n}{nanmin}\PY{p}{(}\PY{n}{Xpreds}\PY{p}{,} \PY{n}{axis}\PY{o}{=}\PY{l+m+mi}{0}\PY{p}{)}\PY{p}{)}
\PY{n+nb}{print}\PY{p}{(}\PY{l+s+s2}{\PYZdq{}}\PY{l+s+s2}{Max: }\PY{l+s+s2}{\PYZdq{}}\PY{p}{,} \PY{n}{np}\PY{o}{.}\PY{n}{nanmax}\PY{p}{(}\PY{n}{Xpreds}\PY{p}{,} \PY{n}{axis}\PY{o}{=}\PY{l+m+mi}{0}\PY{p}{)}\PY{p}{)}
\end{Verbatim}
\end{tcolorbox}

    \begin{Verbatim}[commandchars=\\\{\}]
Mean:  [5.84333333 3.05733333 3.758     ]
Median:  [5.8  3.   4.35]
Standard deviation:  [0.82530129 0.43441097 1.75940407]
Min:  [4.3 2.  1. ]
Max:  [7.9 4.4 6.9]
\end{Verbatim}

    \hypertarget{feature-correlations}{%
\subsection{FEATURE CORRELATIONS}\label{feature-correlations}}

It's useful to know the correlation between various features, as well as
each feature and the predicted label. Feature correlation is useful for
feature selection and understanding the relationship between multiple
variables within a dataset. Correlation is either positive, negative, or
zero. When two features increase simultaneously, they are positively
correlated. When one feature increases while the other decreases, the
features are negatively correlated. Zero correlation is when there is no
relationship between the features. Correlation is on the range -1
(perfect negative correlation) and 1 (pefect positive correlation).\\
We can construct scatter plots of one feature versuses another to
observe linear or nonlinear relationships.

Complete the following set of scatter plots:

\textless{}\center\textgreater{}

    \begin{tcolorbox}[breakable, size=fbox, boxrule=1pt, pad at break*=1mm,colback=cellbackground, colframe=cellborder]
\prompt{In}{incolor}{13}{\hspace{4pt}}
\begin{Verbatim}[commandchars=\\\{\}]
\PY{l+s+sd}{\PYZdq{}\PYZdq{}\PYZdq{}}
\PY{l+s+sd}{Using the scatter plot function, construct plots depicting the}
\PY{l+s+sd}{correlation between all pairings of the selected predictor features}
\PY{l+s+sd}{and between all predictors and the determined target.}
\PY{l+s+sd}{The figure will contain r by r subplots, where r = npredictors + 1.}
\PY{l+s+sd}{Where subplot(i,j) is a scatter plot of the feature i versus feature j.}
\PY{l+s+sd}{When i == j, plot the histogram of feature i instead of a scatter plot.}
\PY{l+s+sd}{We are also interested in the correlation between each of the features }
\PY{l+s+sd}{and the target classification, thus we will combine the predictors matrix}
\PY{l+s+sd}{and the target vector into one large matrix for convenience.}
\PY{l+s+sd}{\PYZdq{}\PYZdq{}\PYZdq{}}
\PY{c+c1}{\PYZsh{} Append the y to the end of the matrix of predictors}
\PY{n}{Xycombo} \PY{o}{=} \PY{n}{np}\PY{o}{.}\PY{n}{append}\PY{p}{(}\PY{n}{Xpreds}\PY{p}{,} \PY{n}{y}\PY{o}{.}\PY{n}{reshape}\PY{p}{(}\PY{o}{\PYZhy{}}\PY{l+m+mi}{1}\PY{p}{,} \PY{l+m+mi}{1}\PY{p}{)}\PY{p}{,} \PY{n}{axis}\PY{o}{=}\PY{l+m+mi}{1}\PY{p}{)}
\PY{c+c1}{\PYZsh{} Append the name \PYZsq{}target\PYZsq{} to the end of the list of predictor names}
\PY{n}{Xycolnames} \PY{o}{=} \PY{n}{predictorNames} \PY{o}{+} \PY{p}{[}\PY{l+s+s1}{\PYZsq{}}\PY{l+s+s1}{target}\PY{l+s+s1}{\PYZsq{}}\PY{p}{]}

\PY{n}{df\PYZus{}Xycombo} \PY{o}{=} \PY{n}{pd}\PY{o}{.}\PY{n}{DataFrame}\PY{p}{(}\PY{n}{data}\PY{o}{=}\PY{n}{Xycombo}\PY{p}{,}\PY{n}{index}\PY{o}{=}\PY{k+kc}{None}\PY{p}{,} \PY{n}{columns}\PY{o}{=}\PY{n}{Xycolnames}\PY{p}{)}
\PY{c+c1}{\PYZsh{} Create the scatter plots}
\PY{n}{fig}\PY{p}{,} \PY{n}{axs} \PY{o}{=} \PY{n}{plt}\PY{o}{.}\PY{n}{subplots}\PY{p}{(}\PY{n}{predictorSize}\PY{o}{+}\PY{l+m+mi}{1}\PY{p}{,} \PY{n}{predictorSize}\PY{o}{+}\PY{l+m+mi}{1}\PY{p}{,} \PY{n}{figsize}\PY{o}{=}\PY{p}{(}\PY{l+m+mi}{15}\PY{p}{,} \PY{l+m+mi}{15}\PY{p}{)}\PY{p}{)}
\PY{n}{fig}\PY{o}{.}\PY{n}{subplots\PYZus{}adjust}\PY{p}{(}\PY{n}{wspace}\PY{o}{=}\PY{o}{.}\PY{l+m+mi}{35}\PY{p}{)}
\PY{k}{for} \PY{n}{f1} \PY{o+ow}{in} \PY{n+nb}{range}\PY{p}{(}\PY{n}{predictorSize}\PY{o}{+}\PY{l+m+mi}{1}\PY{p}{)}\PY{p}{:}
    \PY{k}{for} \PY{n}{f2} \PY{o+ow}{in} \PY{n+nb}{range}\PY{p}{(}\PY{n}{predictorSize}\PY{o}{+}\PY{l+m+mi}{1}\PY{p}{)}\PY{p}{:} 
        \PY{k}{if} \PY{n}{f1} \PY{o}{==} \PY{n}{f2}\PY{p}{:}
            \PY{l+s+s2}{\PYZdq{}}\PY{l+s+s2}{creates a histogram if feature1 and feature2 are equal}\PY{l+s+s2}{\PYZdq{}}
            \PY{n}{axs}\PY{p}{[}\PY{n}{f1}\PY{p}{]}\PY{p}{[}\PY{n}{f2}\PY{p}{]}\PY{o}{.}\PY{n}{hist}\PY{p}{(}\PY{n}{df\PYZus{}Xycombo}\PY{p}{[}\PY{n}{Xycolnames}\PY{p}{[}\PY{n}{f1}\PY{p}{]}\PY{p}{]}\PY{p}{)}
        \PY{k}{else}\PY{p}{:}
            \PY{l+s+s2}{\PYZdq{}}\PY{l+s+s2}{scatter plots the difference between feature1 and feature2}\PY{l+s+s2}{\PYZdq{}}
            \PY{n}{axs}\PY{p}{[}\PY{n}{f1}\PY{p}{]}\PY{p}{[}\PY{n}{f2}\PY{p}{]}\PY{o}{.}\PY{n}{scatter}\PY{p}{(}\PY{n}{df\PYZus{}Xycombo}\PY{p}{[}\PY{n}{Xycolnames}\PY{p}{[}\PY{n}{f1}\PY{p}{]}\PY{p}{]}\PY{p}{,} \PY{n}{df\PYZus{}Xycombo}\PY{p}{[}\PY{n}{Xycolnames}\PY{p}{[}\PY{n}{f2}\PY{p}{]}\PY{p}{]}\PY{p}{)}
        \PY{k}{if} \PY{n}{f1} \PY{o}{==} \PY{n}{predictorSize}\PY{p}{:}
            \PY{n}{axs}\PY{p}{[}\PY{n}{f1}\PY{p}{,} \PY{n}{f2}\PY{p}{]}\PY{o}{.}\PY{n}{set\PYZus{}xlabel}\PY{p}{(}\PY{n}{Xycolnames}\PY{p}{[}\PY{n}{f2}\PY{p}{]}\PY{p}{)}
        \PY{k}{if} \PY{n}{f2} \PY{o}{==} \PY{l+m+mi}{0}\PY{p}{:}
            \PY{n}{axs}\PY{p}{[}\PY{n}{f1}\PY{p}{,} \PY{n}{f2}\PY{p}{]}\PY{o}{.}\PY{n}{set\PYZus{}ylabel}\PY{p}{(}\PY{n}{Xycolnames}\PY{p}{[}\PY{n}{f1}\PY{p}{]}\PY{p}{)}
\end{Verbatim}
\end{tcolorbox}

    \begin{center}
    \adjustimage{max size={0.9\linewidth}{0.9\paperheight}}{output_21_0.png}
    \end{center}
    { \hspace*{\fill} \\}
    
    \hypertarget{images-and-colormaps}{%
\subsection{IMAGES AND COLORMAPS}\label{images-and-colormaps}}

Create a colormap plot of the correlations between the all the
predictors and the target

    \begin{tcolorbox}[breakable, size=fbox, boxrule=1pt, pad at break*=1mm,colback=cellbackground, colframe=cellborder]
\prompt{In}{incolor}{15}{\hspace{4pt}}
\begin{Verbatim}[commandchars=\\\{\}]
\PY{l+s+sd}{\PYZdq{}\PYZdq{}\PYZdq{} PROVIDED}
\PY{l+s+sd}{Generate a figure that plots the a correlation matrix}
\PY{l+s+sd}{as a colormap.}
\PY{l+s+sd}{PARAMS:}
\PY{l+s+sd}{    corrs: matrix of correlations between the features}
\PY{l+s+sd}{    varnames: list of the names of each of the features }
\PY{l+s+sd}{              (e.g. the column names)}
\PY{l+s+sd}{\PYZdq{}\PYZdq{}\PYZdq{}}
\PY{k}{def} \PY{n+nf}{correlationmap}\PY{p}{(}\PY{n}{corrs}\PY{p}{,} \PY{n}{varnames}\PY{p}{)}\PY{p}{:}
    \PY{n}{nvars} \PY{o}{=} \PY{n}{corrs}\PY{o}{.}\PY{n}{shape}\PY{p}{[}\PY{l+m+mi}{0}\PY{p}{]}
    
    \PY{c+c1}{\PYZsh{} create the figure and plot the correlation matrix}
    \PY{n}{fig}\PY{p}{,} \PY{n}{ax} \PY{o}{=} \PY{n}{plt}\PY{o}{.}\PY{n}{subplots}\PY{p}{(}\PY{p}{)}
    \PY{n}{im} \PY{o}{=} \PY{n}{ax}\PY{o}{.}\PY{n}{imshow}\PY{p}{(}\PY{n}{corrs}\PY{p}{,} \PY{n}{cmap}\PY{o}{=}\PY{l+s+s1}{\PYZsq{}}\PY{l+s+s1}{RdBu}\PY{l+s+s1}{\PYZsq{}}\PY{p}{,} \PY{n}{vmin}\PY{o}{=}\PY{o}{\PYZhy{}}\PY{l+m+mi}{1}\PY{p}{,} \PY{n}{vmax}\PY{o}{=}\PY{l+m+mi}{1}\PY{p}{)}
    \PY{n}{cbar} \PY{o}{=} \PY{n}{ax}\PY{o}{.}\PY{n}{figure}\PY{o}{.}\PY{n}{colorbar}\PY{p}{(}\PY{n}{im}\PY{p}{,} \PY{n}{ax}\PY{o}{=}\PY{n}{ax}\PY{p}{)}
    \PY{n}{cbar}\PY{o}{.}\PY{n}{ax}\PY{o}{.}\PY{n}{set\PYZus{}ylabel}\PY{p}{(}\PY{l+s+s2}{\PYZdq{}}\PY{l+s+s2}{Pearson Correlation}\PY{l+s+s2}{\PYZdq{}}\PY{p}{,} \PY{n}{rotation}\PY{o}{=}\PY{o}{\PYZhy{}}\PY{l+m+mi}{90}\PY{p}{,} \PY{n}{va}\PY{o}{=}\PY{l+s+s2}{\PYZdq{}}\PY{l+s+s2}{bottom}\PY{l+s+s2}{\PYZdq{}}\PY{p}{)}
    
    \PY{c+c1}{\PYZsh{} Specify the row and column ticks and labels for the figure}
    \PY{n}{ax}\PY{o}{.}\PY{n}{set\PYZus{}xticks}\PY{p}{(}\PY{n+nb}{range}\PY{p}{(}\PY{n}{nvars}\PY{p}{)}\PY{p}{)}
    \PY{n}{ax}\PY{o}{.}\PY{n}{set\PYZus{}yticks}\PY{p}{(}\PY{n+nb}{range}\PY{p}{(}\PY{n}{nvars}\PY{p}{)}\PY{p}{)}
    \PY{n}{ax}\PY{o}{.}\PY{n}{set\PYZus{}xticklabels}\PY{p}{(}\PY{n}{varnames}\PY{p}{)}
    \PY{n}{ax}\PY{o}{.}\PY{n}{set\PYZus{}yticklabels}\PY{p}{(}\PY{n}{varnames}\PY{p}{)}

    \PY{c+c1}{\PYZsh{} Rotate the tick labels and set their alignment.}
    \PY{n}{plt}\PY{o}{.}\PY{n}{setp}\PY{p}{(}\PY{n}{ax}\PY{o}{.}\PY{n}{get\PYZus{}xticklabels}\PY{p}{(}\PY{p}{)}\PY{p}{,} \PY{n}{rotation}\PY{o}{=}\PY{l+m+mi}{45}\PY{p}{,} \PY{n}{ha}\PY{o}{=}\PY{l+s+s2}{\PYZdq{}}\PY{l+s+s2}{right}\PY{l+s+s2}{\PYZdq{}}\PY{p}{,} \PY{n}{rotation\PYZus{}mode}\PY{o}{=}\PY{l+s+s2}{\PYZdq{}}\PY{l+s+s2}{anchor}\PY{l+s+s2}{\PYZdq{}}\PY{p}{)}

    \PY{c+c1}{\PYZsh{} Loop over data dimensions and create text annotations.}
    \PY{k}{for} \PY{n}{i} \PY{o+ow}{in} \PY{n+nb}{range}\PY{p}{(}\PY{n}{nvars}\PY{p}{)}\PY{p}{:}
        \PY{k}{for} \PY{n}{j} \PY{o+ow}{in} \PY{n+nb}{range}\PY{p}{(}\PY{n}{nvars}\PY{p}{)}\PY{p}{:}
            \PY{n}{text} \PY{o}{=} \PY{n}{ax}\PY{o}{.}\PY{n}{text}\PY{p}{(}\PY{n}{j}\PY{p}{,} \PY{n}{i}\PY{p}{,} \PY{l+s+s2}{\PYZdq{}}\PY{l+s+si}{\PYZpc{}.3f}\PY{l+s+s2}{\PYZdq{}} \PY{o}{\PYZpc{}} \PY{n}{corrs}\PY{p}{[}\PY{n}{i}\PY{p}{,} \PY{n}{j}\PY{p}{]}\PY{p}{,}
                           \PY{n}{ha}\PY{o}{=}\PY{l+s+s2}{\PYZdq{}}\PY{l+s+s2}{center}\PY{l+s+s2}{\PYZdq{}}\PY{p}{,} \PY{n}{va}\PY{o}{=}\PY{l+s+s2}{\PYZdq{}}\PY{l+s+s2}{center}\PY{l+s+s2}{\PYZdq{}}\PY{p}{,} \PY{n}{color}\PY{o}{=}\PY{l+s+s2}{\PYZdq{}}\PY{l+s+s2}{k}\PY{l+s+s2}{\PYZdq{}}\PY{p}{)}
\PY{c+c1}{\PYZsh{} END DEF correlationmap}
            

\PY{l+s+sd}{\PYZdq{}\PYZdq{}\PYZdq{} TODO}
\PY{l+s+sd}{Compute the Pearson correlation between the columns of Xycombo using}
\PY{l+s+sd}{the numpy function corrcoef(). The corrcoef() function performs the }
\PY{l+s+sd}{the pairwise correlation on the rows of a matrix, thus you will need to}
\PY{l+s+sd}{transpose the input.}
\PY{l+s+sd}{\PYZdq{}\PYZdq{}\PYZdq{}} 

\PY{l+s+s2}{\PYZdq{}}\PY{l+s+s2}{transposes Xycombo then is used in np.corrcoef}\PY{l+s+s2}{\PYZdq{}}
\PY{n}{Xycorrs} \PY{o}{=} \PY{n}{np}\PY{o}{.}\PY{n}{corrcoef}\PY{p}{(}\PY{n}{np}\PY{o}{.}\PY{n}{transpose}\PY{p}{(}\PY{n}{Xycombo}\PY{p}{)}\PY{p}{)}

\PY{l+s+sd}{\PYZdq{}\PYZdq{}\PYZdq{} TODO}
\PY{l+s+sd}{Call the function defined above, correlationmap(), to generate a }
\PY{l+s+sd}{colormap plot of the correlations between columns of the Xycombo matrix}
\PY{l+s+sd}{\PYZdq{}\PYZdq{}\PYZdq{}}

\PY{l+s+s2}{\PYZdq{}}\PY{l+s+s2}{calling correlationmap by passing Xycorrs and Xycolnames}\PY{l+s+s2}{\PYZdq{}}
\PY{n}{correlationmap}\PY{p}{(}\PY{n}{Xycorrs}\PY{p}{,} \PY{n}{Xycolnames}\PY{p}{)}
\end{Verbatim}
\end{tcolorbox}

    \begin{center}
    \adjustimage{max size={0.9\linewidth}{0.9\paperheight}}{output_23_0.png}
    \end{center}
    { \hspace*{\fill} \\}
    
    \begin{tcolorbox}[breakable, size=fbox, boxrule=1pt, pad at break*=1mm,colback=cellbackground, colframe=cellborder]
\prompt{In}{incolor}{ }{\hspace{4pt}}
\begin{Verbatim}[commandchars=\\\{\}]

\end{Verbatim}
\end{tcolorbox}


    % Add a bibliography block to the postdoc
    
    
    
    \end{document}
